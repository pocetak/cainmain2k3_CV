%%%%%%%%%%%%%%%%%%%%%%%%%%%%%%%%%%%%%%%%%
% Medium Length Professional CV
% LaTeX Template
%
% This template has been downloaded from:
% http://www.LaTeXTemplates.com
%
% Original author:
% Trey Hunner (http://www.treyhunner.com/)
%
% Important note:
% This template requires the resume.cls file to be in the same directory as the
% .tex file. The resume.cls file provides the resume style used for structuring the
% document.
%
%%%%%%%%%%%%%%%%%%%%%%%%%%%%%%%%%%%%%%%%%

%----------------------------------------------------------------------------------------
%	PACKAGES AND OTHER DOCUMENT CONFIGURATIONS
%----------------------------------------------------------------------------------------

\documentclass{resume} % Use the custom resume.cls style

\usepackage[left=0.75in,top=0.4in,right=0.75in,bottom=0.4in]{geometry} % Document margins

\name{} % Your name


\begin{document}

\begin{tabular*}{7in}{l@{\extracolsep{\fill}}r}
 & \\
& \\
\end{tabular*}

\hline

%----------------------------------------------------------------------------------------
%	OBRAZOVANJE
%----------------------------------------------------------------------------------------

\begin{rSection}{OBRAZOVANJE}

\begin{rSubsection}{Magistar in\v{z}enjer elektrotehnike i informacijske tehnologije}{srpanj 2010. - srpanj 2012.}{Sveu\v{c}ili\v{s}te u Zagrebu, Fakultet elektrotehnike i ra\v{c}unarstva, smjer: elektroenergetika}{}
%\item Diplomski rad: Prora\v{c}un akcidenta gubitak tereta za NE Kr\v{s}ko programom TRACE
\end{rSubsection}

\begin{rSubsection}{Baccalaureus elektrotehnike i informacijske tehnologije}{srpanj 2007. - srpanj 2010.}{Sveu\v{c}ili\v{s}te u Zagrebu, Fakultet elektrotehnike i ra\v{c}unarstva, smjer: elektroenergetika}{}
%\item Zavr\v{s}ni rad: Programski alat za naprednu analizu RELAP5 prora\v{c}una
\end{rSubsection}

\end{rSection}

%----------------------------------------------------------------------------------------
%	PROJEKTI
%----------------------------------------------------------------------------------------

\begin{rSection}{Projekti}

\begin{rSubsection}{Prora\v{c}un akcidenta gubitak tereta za NE Kr\v{s}ko programom TRACE}{}{}{}
\item Oblikovane hidrauli\v{c}ke komponente, toplinske strukture i model neutronske kinetike na osnovu parametara iz RELAP5 nodalizacije NE Kr\v{s}ko
\item Obavljen prora\v{c}un ravnote\v{z}nog stanja i prora\v{c}un tranzijenta za pojedine komponente, verifikacija rezultata pomo\'{c}u rezultata RELAP5 prora\v{c}una
\end{rSubsection}

%\begin{rSubsection}{Aplikacija za prikaz koncentracija ksenona, joda i samarija u nuklearnom reaktoru}{}{}{}
%\item Implementiran odabir vremenskih intervala unutar kojih je mogu\'{c}e definirati snagu reaktora za svaki pojedini interval te odabrati brzinu simulacije 
%\item Dizajnirano su\v{c}elje koje omogu\'{c}uje grafi\v{c}ki prikaz snage nuklearnog reaktora u kombinaciji sa koncentracijom ksenona, samarija i joda te prikaz gradijenata iz bilanci koncentracije ksenona i joda
%\end{rSubsection}

\begin{rSubsection}{Programski alat za naprednu analizu RELAP5 prora\v{c}una}{}{}{}
\item Razvijen dodatni modul za inicijalnu aplikaciju koji omogu\'{c}ava definiranje 2D geometrije problema i izradu odgovaraju\'{c}e vizualizacijske maske za prikazivanje vrijednosti izra\v{c}unatih veli\v{c}ina
\item Integrirana opcija direktnog u\v{c}itavanja podataka iz RELAP5 restart-plot datoteke te koloriranja vizualizacijske maske za zadanu varijablu u ovisnosti o odabranom rasponu vrijednosti (global/local)
\end{rSubsection}

%\begin{rSubsection}{Prora\v{c}un neplaniranog otvaranja PORV ventila tla\v{c}nika za NE Kr\v{s}ko ciklus 24 na punoj snazi}{}{}{}
%\item Na osnovu postoje\'{c}e RELAP5 nodalizacije NE Kr\v{s}ko obavljen prora\v{c}un za dobivanje parametara stacionarnog stanja elektrane u uvjetima po\v{c}etka odgora te pune snage reaktora
%\item Modificiran generi\v{c}ki input za tranzijent na osnovu uvjeta postavljenih tranzijentom neplaniranog otvaranja rasteretnog ventila tla\v{c}nika u vremenu trajanja prora\v{c}una od 600s
%\end{rSubsection}

\begin{rSubsection}{eSTUDENT natjecanje - Pobolj\v{s}anje energetske u\v{c}inkovitosti tvornice vapna u Li\v{c}kom Le\v{s}\'{c}u}{}{}{}
\item Predlo\v{z}eno kalibriranje parametara prema novom izra\v{c}unu varijabli koji definiraju proces te prilagodba instrumenata na izra\v{c}unatu brzinu prolaska sirovine kroz kotao i brzinu usisa zraka u kotao 
\item Prora\v{c}un u\v{s}teda zamjenom postoje\'{c}ih rasvjetnih tijela s novima u tvornici i popratnim objektima
\end{rSubsection}

\end

\end{rSection}

%----------------------------------------------------------------------------------------
%	RADNO ISKUSTVO
%----------------------------------------------------------------------------------------

\begin{rSection}{Radno iskustvo}

\begin{rSubsection}{Ispitni administrator PISA istra\v{z}ivanja}{velja\v{c}a 2012. - travanj 2012.}{Nacionalni centar za vanjsko vrednovanje obrazovanja, Zagreb}{}
\item Putovanja tri puta tjedno u gradove \v{s}irom Hrvatske (cca. 5000 km)
\item Pripremanje 10-15 prijenosnih ra\v{c}unala za testiranje, upoznavanje u\v{c}enika s na\v{c}inom testiranja, pru\v{z}anje tehni\v{c}ke podr\v{s}ke tijekom testiranja
\item Obrada i backup rezultata nakon zavr\v{s}etka testiranja 
\end{rSubsection}

\end{rSection}

%----------------------------------------------------------------------------------------
%	TEHNIČKE VJEŠTINE
%----------------------------------------------------------------------------------------

\begin{rSection}{Ra\v{c}unalne vje\v{s}tine}
\vspace*{0.4em}
\begin{tabular}{ @{} >{\bfseries}l @{\hspace{6ex}} l }
Programski alati & SNAP (RELAP5, TRACE), LabVIEW, MS Office, Eplan, AutoCAD \\
Programski jezici & Python, C \latex \\
\end{tabular}

\end{rSection}

%----------------------------------------------------------------------------------------
%	Strani Jezici
%----------------------------------------------------------------------------------------

\begin{rSection}{Strani jezici}
\begin{rSubsection}{Engleski}{}{}{}
\item Polo\v{z}en University of Cambridge: First Certificate in English - level C1
\end{rSubsection}
\end{rSection}

%----------------------------------------------------------------------------------------

\end{document}
